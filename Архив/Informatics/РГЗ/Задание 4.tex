\documentclass[12pt, oneside, a4paper]{article}
\usepackage[T2A]{fontenc}
\usepackage[utf8]{inputenc}
\usepackage[russian, english]{babel}
\usepackage{graphicx}
\usepackage{amsmath}
\usepackage{caption}
\usepackage{multirow}
\usepackage{indentfirst}
\usepackage{hyperref}
\usepackage{color}
\usepackage{setspace}
\usepackage[left=3cm, right=1.5cm, top=2cm, bottom=2cm]{geometry}
\usepackage{enumerate}
\linespread{1.5}
\setlength{\parskip}{0cm}

\begin{document}

\thispagestyle{empty}
\begin{center}
\large{\textsc{МИНИСТЕРСТВО НАУКИ И ВЫСШЕГО ОБРАЗОВАНИЯ РОССИЙСКОЙ ФЕДЕРАЦИИ\\
ФЕДЕРАЛЬНОЕ ГОСУДАРСТВЕННОЕ БЮДЖЕТНОЕ ОБРАЗОВАТЕЛЬНОЕ УЧРЕЖДЕНИЕ ВЫСШЕГО
ОБРАЗОВАНИЯ \\}}
\textbf{«БЕЛГОРОДСКИЙ ГОСУДАРСТВЕННЫЙ
ТЕХНОЛОГИЧЕСКИЙ УНИВЕРСИТЕТ им. В. Г. ШУХОВА»
(БГТУ им. В.Г. Шухова)}\\[1mm]
Кафедра программного обеспечения вычислительной техники и автоматизированных
систем\\[4.5cm]
\textbf{\huge Индивидуальное домашнее задание} \\
\large по дисциплине: Информатика\\
\large тема: «Защита~информации.~Авторские~права~на
программное~обеспечение»\\[3.5cm]
\end{center}
\begin{flushright}
\begin{minipage}{.45\textwidth}
Выполнил: ст. группы ПВ-211\\
Чувилко Илья Романович\\[3mm]
Проверила:\\
Бондаренко Татьяна Владимировна
\end{minipage}
\end{flushright}
\vfill
\begin{center}
\large Белгород 2021 г.
\end{center}
\thispagestyle{empty}
\newpage
\tableofcontents
\begin{sloppypar}
\setcounter{page}{2}

\newpage


\section{Введение}
\textbf {Авторское право} — институт гражданского права, регулирующий правоотношения, связанные с созданием и использованием (изданием, исполнением, показом и т. д.) произведений науки, литературы или искусства, то есть объективных результатов творческой деятельности людей в этих областях. \textbf {Программы для ЭВМ} и \textbf {базы данных} также охраняются авторским правом. Они приравнены к литературным произведениям и сборникам, соответственно. 
 
 Английский термин копирайт © (англ. \textbf{copyright}, от \textbf{«копировать»} и \textbf{«право»}) в английском языке обозначает имущественное авторское право, то есть право копировать, воспроизводить.\\
 
 \textbf{Патент} - это набор исключительных прав, предоставляемых государством патентообладателю на ограниченный период времени, обычно 20 лет. Эти права предоставляются заявителям на получение патента в обмен на раскрытие ими изобретений. После выдачи патента в данной стране никто не может производить, использовать, продавать или импортировать / экспортировать заявленное изобретение в этой стране без разрешения патентообладателя. Разрешение, если оно предоставляется, обычно имеет форму лицензии, условия которой устанавливаются патентообладателем: оно может быть бесплатным или в обмен на выплату роялти или единовременного сбора.\\
 
 \textbf{Безлицензионное программное обеспечение} - это \textit{компьютерное программное} обеспечение, которое явно не является общественным достоянием, но авторы, по всей видимости, намереваются бесплатное использование, модификация, распространение и распространение модифицированного программного обеспечения, аналогично свободам, определенным для свободного программного обеспечения. Поскольку автор программного обеспечения не изложил в явной форме условия лицензии, программное обеспечение технически защищено авторским правом в соответствии с Бернской конвенцией и, как таковое, является собственностью.\\
 
\newpage


\section{История авторского права в области программного обеспечения}
Исторически компьютерные программы не были эффективно защищены авторскими правами, потому что компьютерные программы \textbf{не рассматривались} как фиксированный, материальный объект: объектный код рассматривался как утилитарный товар, созданный из исходного кода, а не как творческая работа. Из-за отсутствия прецедента этот результат был достигнут при принятии решения о том, как обращаться с авторскими правами на компьютерные программы. Бюро регистрации авторских прав попыталось классифицировать компьютерные программы, проводя аналогию: чертежи моста и результирующего моста по сравнению с исходным кодом программы и полученным исполняемым объектным кодом. Эта аналогия заставила Бюро регистрации авторских прав выдавать свидетельства об авторских правах в соответствии со своим «Правилом сомнения».

В \textbf{1974} году была создана Комиссия по новому технологическому использованию произведений, охраняемых авторским правом ( CONTU ). CONTU решила, что «\textbf{компьютерные программы}, в той степени, в которой они воплощают оригинальное творение автора, являются надлежащим объектом авторского права». В \textbf{1980} году Конгресс США добавил определение «компьютерной программы» в 17 USC  § 101 и внес поправки в 17 USC  § 117, чтобы позволить владельцу программы сделать еще одну копию или адаптацию для использования на компьютере.

\newpage


\section{Патент на программное обеспечение}
\textbf{Патент программного обеспечения} является патентом на куске программного обеспечения, такие как \textit{компьютерная программа, библиотеки, пользовательский интерфейс,} или \textit{алгоритм}. \\


В \textbf{Российской Федерации} согласно статье 1350 ГКРФ изобретениями не считаются: 
\begin{enumerate}
\item открытия; 
\item научные теории и математические методы;
\item решения, касающиеся только внешнего вида продуктов и направленные на удовлетворение эстетических потребностей;
\item правила и методы игр, интеллектуальной или экономической деятельности;
\item компьютерные программы;
\item решения, заключающиеся только в представлении информации.
\end{enumerate}% конец помеченного списка.
 
Однако статья предусматривает, что патентоспособность этих объектов исключается только в том случае, если заявка на выдачу патента на изобретение касается этих объектов как таковых.\\
\subsection{Очевидность}
Распространенное возражение против патентов на программы состоит в том, что они относятся к тривиальным изобретениям. Утверждается, что патент на изобретение, которое многие люди могут легко разработать независимо друг от друга, не следует выдавать, поскольку это препятствует развитию. 

В разных странах по-разному подходят к вопросу об изобретательском уровне и неочевидности патентов на программы. Европа использует «тест изобретательского уровня»; см. требования об изобретательском уровне в Европе и, например, T 258/03.\\ 


\subsection{Перекрытие с авторским правом}
\textbf{Патент} и \textbf{защита авторских прав} представляют собой два разных средства правовой защиты, которые могут охватывать один и тот же объект, например \textit{компьютерные программы}, поскольку каждое из этих двух средств защиты служит своей цели. \textit{Программное обеспечение} охраняется как произведения литературы в соответствии с Бернской конвенцией. Это позволяет создателю \textbf{предотвратить} копирование программы другим объектом, и, как правило, нет необходимости регистрировать код для \textbf{защиты авторских прав}. 

\textbf{Патенты}, с другой стороны, дают их владельцам право запрещать другим \textit{использовать технологию}, определенную в формуле изобретения, даже если технология была разработана независимо и не было задействовано копирование программного обеспечения или программного кода. Фактически, одно из самых последних решений ЕПВ проясняет это различие, заявляя, что программное обеспечение является патентоспособным, потому что это, по сути, только \textit{технический метод}, выполняемый на компьютере, который следует отличать от самой программы для выполнения метода, при этом программа является просто выражение метода и, следовательно, \textbf{защищено авторским правом}.

\subsection{Программное обеспечение с открытым исходным кодом}
Существует сильная нелюбовь в сообществе свободного программного обеспечения к патентам на программное обеспечение. Во многом это было вызвано тем, что свободное программное обеспечение или проекты с открытым исходным кодом \textbf{прекращались}, когда владельцы патентов, охватывающих аспекты проекта, требовали лицензионных сборов, которые проект не мог заплатить или не желал платить, или предлагали лицензии на условиях, на которых проект был не желает принять или не может принять, потому что это противоречит используемой лицензии свободных программ.

\newpage


\section{Нарушение авторского права}
\textbf{Нарушение авторских прав} (иногда именуемое \textbf{пиратством}) -- это использование работ, защищенных законом об \textit{авторском праве}, без разрешения для использования там, где такое разрешение требуется, тем самым нарушая определенные исключительные права, предоставленные правообладателю, такие как право на воспроизведение, распространение, отображать или выполнять защищенную работу, или создавать производные работы . Обладателем авторских прав обычно является создатель произведения, издатель или другое предприятие, которому переданы авторские права. Правообладатели регулярно прибегают к юридическим и технологическим мерам для предотвращения и наказания за нарушение авторских прав.\\

Споры о нарушении авторских прав обычно разрешаются путем прямых \textit{переговоров, уведомления и удаления} или \textit{судебного разбирательства} в гражданском суде. Грубое или крупномасштабное коммерческое нарушение, особенно когда оно связано с подделкой, иногда преследуется через систему уголовного правосудия. Изменение общественных ожиданий, развитие цифровых технологий и растущий охват Интернета привели к такому широко распространенному \textit{анонимному} нарушению прав, что отрасли, зависящие от авторского права, теперь меньше сосредотачиваются на преследовании лиц, которые ищут и делятся защищенным авторским правом контентом в Интернете, и больше на расширении законодательства об авторском праве признавать и наказывать в качестве косвенных нарушителей \textbf{поставщиков услуг} и \textbf{дистрибьюторов программного обеспечения}, которые, как утверждается, способствуют и поощряют отдельные акты нарушения прав со стороны других лиц. 

\subsection{Ограничения}
Закон об авторском праве \textit{не предоставляет авторам и издателям абсолютный контроль} над использованием их работ. Защищены только определенные виды работ и виды использования; только несанкционированное использование охраняемых произведений может считаться нарушением авторских прав.

\subsection{Существующие законы}
В большинстве стран защита авторских прав распространяется на \textbf{авторов произведений}. В странах с законодательством об авторском праве обеспечение соблюдения авторских прав, как правило, является обязанностью правообладателя. Однако в некоторых юрисдикциях также предусмотрены уголовные наказания за нарушение авторских прав. 

\subsection{Законность скачивания}
В некоторой степени закон об авторском праве в некоторых странах \textit{разрешает} загружать защищенный авторским правом контент для личного некоммерческого использования. Примеры включают Канаду и государства-члены Европейского союза (ЕС), такие как \textbf{Польша} и \textbf{Нидерланды}.

\subsection{Законность загрузки}
Хотя загрузка или другое частное копирование иногда разрешено, публичное распространение - путем загрузки или иного предложения поделиться контентом, защищенным авторским правом - остается незаконным в большинстве, если не во всех странах. Например, в Канаде, даже несмотря на то, что когда-то было законно скачивать любой защищенный авторским правом файл, пока он был для некоммерческого использования, распространение защищенных авторским правом файлов (например, путем загрузки их в P2P-сеть) по-прежнему было незаконным.

\subsection{Ответственность онлайн-посредника}
Вопрос о том, несут ли интернет-посредники\footnote{Под интернет-посредниками раньше понимались поставщики интернет-услуг (ISP). Однако вопросы ответственности возникли и в отношении других посредников инфраструктуры Интернета, включая поставщиков магистральных сетей Интернета, кабельные компании и поставщиков мобильной связи} ответственность за нарушение авторских прав своими пользователями, является предметом дебатов и судебных разбирательств в ряде стран.

\newpage


\section{Список используемой литературы}
\begin{itemize}
 \item Авторское право.\\
 URL: https://ru.wikipedia.org/wiki/Авторское\_право
 
 \item Авторское право на программное обеспечение - Software copyright\\
 URL: https://ru.abcdef.wiki/wiki/Software\_copyright
 
 \item Безлицензионное программное обеспечение - License-free software.\\
 URL: https://ru.abcdef.wiki/wiki/License-free\_software
 
  \item Патент на программное обеспечение - Software patent.\\
 URL: https://ru.abcdef.wiki/wiki/Software\_patent
 
   \item Нарушение авторского права - Copyright infringement.\\
 URL: https://ru.abcdef.wiki/wiki/Copyright\_infringement

\end{itemize}
\end{sloppypar}
\end{document}